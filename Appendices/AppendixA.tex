% Appendix A

\chapter{Configuration - config.json - Experiment 3} % Main appendix title
\label{appendix:config.json} % For referencing this appendix elsewhere, use \ref{AppendixA}
\lhead{Appendix A. \emph{Configuration - config.json - Experiment 3}} % This is for the header on each page - perhaps a shortened title

Here you find the configuration as used during Experiment 3. More precisely test 3 of experiment 3. It uses a total of 4 workers (10 connections each) assigned to 2 groups. Group 0 having 1 worker and group 1 with 3 workers.
\newline
The format is in JSON.\cite{wiki:json}
\begin{lstlisting}[language=JavaScript]
{
  "group": "the group we are crawling for. defined by argv[1]",
  "block": "the block we are crawling for. defined by argv[2]",
  "mapper": {
    "rules": [
      {"hostname": "bs.wikipedia.org", "group": 0 },
      {"hostname": "bn.wikipedia.org", "group": 1 }
    ],
    "groups": [
      {"group": 0, "blocks": 1},
      {"group": 1, "blocks": 3}
    ]
  },
  "fetcher": {
    "instances": 10,
    "request": {
        "jar": true,
        "timeout": 30000,
        "followRedirect": true,
        "maxRedirects": 3,
        "headers": {
          "User-Agent": "crawl.js v0.0.1"
        }
    }
  },
  "extractor": "parser",
  "queues": {
    "local": {
      "type": "memory",
      "options": {
        "limit": 10000
      }
    },
    "remote": {
      "type": "redis",
      "options": {
        "flushInterval": 10000,
        "host": "redis0",
        "port": "6379"
      }
    }
  },
  "seen": {
    "host": "redis0",
    "port": "6379"
  },
  "dispatcher": {
    "acceptPattern": "http://(bs|bn).wikipedia.org/articles"
  },
  "store": {
    "type": "dummy",
    "options": {}
  },
  "robo": {
    "limit": 100
  }
}
\end{lstlisting}