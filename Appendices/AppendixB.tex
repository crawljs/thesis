% Appendix A

\chapter{Worker hardware details (lshw)} % Main appendix title
\label{appendix:worker} % For referencing this appendix elsewhere, use \ref{AppendixA}
\lhead{Appendix A. \emph{Worker hardware details}} % This is for the header on each page - perhaps a shortened title

Thats the output (stripped down) of the command \emph{lshw} from one of our VMs used throughout the experiments. With such hardware we achieved around 250 pages / second.
\newline
\begin{lstlisting}[language=Bash]
sudo lshw -c cpu -c memory -c network
  *-cpu
       description: CPU
       product: QEMU Virtual CPU version 1.0
       vendor: Intel Corp.
       physical id: 401
       bus info: cpu@0
       slot: CPU 1
       size: 2GHz
       capacity: 2GHz
       width: 64 bits
  *-memory
       description: System Memory
       physical id: 1000
       size: 1GiB
       capacity: 1GiB
     *-bank
          description: DIMM RAM
          physical id: 0
          slot: DIMM 0
          size: 1GiB
          width: 64 bits
  *-network
       description: Ethernet interface
       product: RTL-8139/8139C/8139C+
       vendor: Realtek Semiconductor Co., Ltd.
       physical id: 3
       bus info: pci@0000:00:03.0
       logical name: eth0
       version: 20
       serial: 02:00:c0:a8:62:09
       size: 100Mbit/s
       capacity: 100Mbit/s
       width: 32 bits
       clock: 33MHz
  *-memory
       description: RAM memory
       product: Virtio memory balloon
       vendor: Red Hat, Inc
       physical id: 4
       bus info: pci@0000:00:04.0
       version: 00
       width: 32 bits
       clock: 33MHz (30.3ns)
       capabilities: bus_master
       configuration: driver=virtio-pci latency=0
       resources: irq:11 ioport:c120(size=32)
\end{lstlisting}