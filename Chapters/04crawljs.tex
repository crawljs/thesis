
\chapter{Crawl.js} % Main chapter title
In this chapter we describe how crawl.js works in detail. First we give an overview of the architecture and introduce some terms we will use througout the next sections. 
\label{Chapter4} 
\lhead{Chapter 4. \emph{Crawl.js}} 

\section{System overview}
To understand better how crawl.js works we will start with an illustration.
\subsection{A Node}
Dummy
\subsection{Mapper: The load balancer}
Dummy
\subsection{Storage}
Dummy

\section{Distributed}
Dummy
\subsection{Static allocation}
Dummy
\subsection{Dynamic allocation}
Dummy
\subsection{Synchronization}
Dummy
\subsection{Load-Balancing}
Dummy

\section{Storage}
Dummy
\subsection{Riak}
Dummy
\subsection{Redis}
Dummy
\subsection{Combined}
Dummy

\section{Url}
Dummy
\subsection{Normalization}
Dummy

\section{A Node: The software}
Dummy
\subsection{Architecture}
Dummy
\subsection{Node.js}
Dummy
\subsection{github}
Dummy
\subsection{Performance}
Dummy
\subsection{Storage}
Dummy

\section{Faut Tolerance}
Dummy

