% Chapter Template

\chapter{Related work} % Main chapter title

The active research area of web crawling in general is big. There are a lot of topics like selection policy, re-visit policy, url-normalization, .. In this thesis we want to focus on the work that has been done regarding architectures of web crawlers. Quite a few have actually tried to build a scalable, parallel and distributed web crawler. The following sections present the different papers organized by their idea.

\label{Chapter2} % Change X to a consecutive number; for referencing this chapter elsewhere, use \ref{ChapterX}

\lhead{Chapter 2. \emph{Related work}} % Change X to a consecutive number; this is for the header on each page - perhaps a shortened title

\section{Sequential crawlers}
One page is downloaded at the time. Nothing happens in parallel. Theses kind of web crawlers were used in the very beginning of the web. As the number of web pages grew the need to download web pages at a higher rate was obvious.



%----------------------------------------------------------------------------------------
%	SECTION 1
%----------------------------------------------------------------------------------------
\section{Parallel crawlers}





\section{Centralized parallel crawlers}

Centralized crawlers have one part in their system that has the following properties:


%-----------------------------------
%	SUBSECTION 2
%-----------------------------------

\subsection{...}

%----------------------------------------------------------------------------------------
%	SECTION 2
%----------------------------------------------------------------------------------------

\section{Decentralized(Peer-to-Peer) parallel crawlers}

This approach is totally different to the centralized..

%-----------------------------------
%	SUBSECTION 1
%-----------------------------------
\subsection{UbiCrawler: a scalable fully distributed Web crawler}

