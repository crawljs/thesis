% Chapter Template

\chapter{Introduction} % Main chapter title

\label{Chapter1} % Change X to a consecutive number; for referencing this chapter elsewhere, use \ref{ChapterX}

\lhead{Chapter 1. \emph{Introduction}} % Change X to a consecutive number; this is for the header on each page - perhaps a shortened title

These days, information is very important. Not only information itself, but the \emph{access} to it is central. Most information is found there, where everyone contributes to - the world wide web. Here we have many creators (authors, engineers, teachers, ...) on many different sites and this means: fragmented information. Even if a certain piece of information is there we can't \emph{access} it, because we don't know where to find it. Thats unsatisfying and thats why search engines \footnote{\url{http://en.wikipedia.org/wiki/Web_search_engine}} are so important to us.
\newline
Unfortunately, today, only a few big companies are able to deal with both the engineering challenge and the resources (hardware) it requires, to collect all the information and give us \emph{access} to it. To something \emph{we} created notably. Even more, we don't really know \emph{how} they do it. There is a lot of research in determining if a given page is important or not\cite{page_importance1}\cite{page_importance2} (an indicator for information quality). Or  


\section{Node.js}

http://nodejs.org

\section{Outline}
In Chapter~\ref{Chapter2} we present the related work that has been done in this field and where crawl.js fits in.

